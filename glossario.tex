
%**************************************************************
% Acronimi
%**************************************************************
\renewcommand{\acronymname}{Acronimi e abbreviazioni}

\newacronym[description={\glslink{apig}{Application Program Interface}}]
    {api}{API}{Application Program Interface}
    
\newacronym[description={\glslink{ictg}{Information and Communication Technologies}}]
	{ict}{ICT}{Information and Communication Technologies}
	
\newacronym[description={\glslink{pecg}{Posta Elettronica Certificata}}]
	{pec}{PEC}{Posta Elettronica Certificata}
	
\newacronym[description={\glslink{pocg}{proof of concept}}]
	{poc}{poc}{proof of concept}

\newacronym[description={\glslink{spidg}{Sistema Pubblico d'Identità Digitale}}]
	{spid}{SPID}{Sistema Pubblico d'Identità Digitale}

\newacronym[description={\glslink{umlg}{Unified Modeling Language}}]
    {uml}{UML}{Unified Modeling Language}

%**************************************************************
% Glossario
%**************************************************************
\renewcommand{\glossaryname}{Glossario}

\newglossaryentry{apig}
{
    name=\glslink{api}{API},
    text=Application Program Interface,
    sort=api,
    description={in informatica con il termine \emph{Application Programming Interface (API)} (ing. interfaccia di programmazione di un'applicazione) si indica ogni insieme di procedure disponibili al programmatore, di solito raggruppate a formare una collezione di strumenti specifici per l'espletamento di un determinato compito all'interno di un certo programma. La finalità è ottenere un'astrazione, di solito tra l'hardware e il programmatore o tra software a basso e quello ad alto livello semplificando così il lavoro di programmazione}
}

\newglossaryentry{blockchaing}
{
	name=Blockchain,
	text=blockchain,
	sort=blockchain,
	description={in informatica con il termine \emph{blockchain} (ing. catena di blocchi) si indica ...}
}

\newglossaryentry{ictg}
{
	name=\glslink{ict}{ICT},
	text=Information and Communication Technologies,
	sort=ict,
	description={in informatica con il termine \emph{Information and Communication Technologies (ICT)} (ing. tecnologie di informazione e comunicazione) si indica ogni insieme di tecnologie riguardanti i sistemi integrati di telecomunicazione (linee di comunicazione cablate e senza fili), i \textit{computer}, le tecnologie audio-video e relativi \textit{software}, che permettono agli utenti di creare, immagazzinare e scambiare informazioni}
}

\newglossaryentry{open-sourceg}
{
	name=Open-source,
	text=open-source,
	sort=open-source,
	description={in informatica con il termine \emph{open-source} (ing. sorgente libero) si indica ...}
}

\newglossaryentry{pecg}
{
	name=\glslink{pec}{PEC},
	text=Posta Elettronica Certificata,
	sort=pec,
	description={in informatica con il termine \emph{Posta Elettronica Certificata (PEC)} si indica il sistema di posta elettronica che fornisce al mittente ricevuta con valore legale dell’invio e della ricezione di messaggi informatici. La \textit{PEC} è quindi un sistema in grado di certificare l'identità (digitale) di un soggetto che ne fa utilizzo, garantendo inoltre il non ripudio per tutte le operazioni da esso svolte}
}

\newglossaryentry{pocg}
{
	name=\glslink{poc}{POC},
	text=proof of concept,
	sort=poc,
	description={in informatica con il termine \emph{proof of concept (poc)} si indica ...}
}

\newglossaryentry{spidg}
{
	name=\glslink{spid}{SPID},
	text=Sistema Pubblico d'Identità Digitale,
	sort=spid,
	description={con il termine \emph{Sistema Pubblico d'Identità Digitale (SPID)} si indica la piattaforma tramite cui le pubbliche amministrazioni consentono ai cittadini e alle imprese l'accesso in rete ai propri servizi tramite un unico sistema di \textit{password}. Questo sistema consente quindi di identitificare univocamente ogni soggetto che ne fa utilizzo, fornendogli un'identità digitale che lo rappresenta e che gli permette di usufruire di tutti quei servizi che richiedono un'autenticazione tramite documento d'identificazione}
}

\newglossaryentry{umlg}
{
    name=\glslink{uml}{UML},
    text=UML,
    sort=uml,
    description={in ingegneria del \textit{software} l'\emph{Unified Modeling Language (UML)} (ing. linguaggio di modellazione unificato) è un linguaggio di modellazione e specifica basato sul paradigma \textit{object-oriented}. L'\emph{UML} svolge un'importantissima funzione di ``lingua franca'' nella comunità della progettazione e programmazione a oggetti. Gran parte della letteratura di settore usa tale linguaggio per descrivere soluzioni analitiche e progettuali in modo sintetico e comprensibile a un vasto pubblico}
}
