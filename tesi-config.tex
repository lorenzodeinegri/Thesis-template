%**************************************************************
% file contenente le impostazioni della tesi
%**************************************************************

%**************************************************************
% Frontespizio
%**************************************************************

% Autore
\newcommand{\myName}{Lorenzo Dei Negri}

% Titolo                               
\newcommand{\myTitle}{Identità digitali per scambio e firma di documenti fiscali tramite blockchain}

% Tipo di tesi                   
\newcommand{\myDegree}{Tesi di laurea triennale}

% Università             
\newcommand{\myUni}{Università degli Studi di Padova}

% Facoltà       
\newcommand{\myFaculty}{Corso di Laurea in Informatica}

% Dipartimento
\newcommand{\myDepartment}{Dipartimento di Matematica "Tullio Levi-Civita"}

% Titolo del relatore
\newcommand{\profTitle}{Prof.}

% Relatore
\newcommand{\myProf}{Massimo Marchiori}

% Luogo
\newcommand{\myLocation}{Padova}

% Anno accademico
\newcommand{\myAA}{2019-2020}

% Data discussione
\newcommand{\myTime}{Luglio 2020}

% Nome azienda ospitante
\newcommand{\myCompany}{2C Solution}

% Ragione sociale azienda ospitante
\newcommand{\companyTitle}{S.r.l.}

% Tutor aziendale
\newcommand{\myTutor}{Andrea Bisello}

%**************************************************************
% Impostazioni di impaginazione
% see: http://wwwcdf.pd.infn.it/AppuntiLinux/a2547.htm
%**************************************************************

\setlength{\parindent}{14pt}   							% larghezza rientro della prima riga
\setlength{\parskip}{0pt}   							% distanza tra i paragrafi

%**************************************************************
% Impostazioni di biblatex
%**************************************************************

\bibliography{bibliografia} 							% database di biblatex 

\defbibheading{bibliography} {
    \cleardoublepage
    \phantomsection 
    \addcontentsline{toc}{chapter}{\bibname}
    \chapter*{\bibname\markboth{\bibname}{\bibname}}
}

\setlength\bibitemsep{1.5\itemsep}						% spazio tra entry

\DeclareBibliographyCategory{articoli}
\DeclareBibliographyCategory{manuali}
\DeclareBibliographyCategory{web}

\addtocategory{articoli}{article:blockchain4innovation}

\addtocategory{articoli}{article:blockchain-actors}
\addtocategory{articoli}{article:blockchain-app}
\addtocategory{articoli}{article:blockchain-decentralization}
\addtocategory{articoli}{article:blockchain-development}
\addtocategory{articoli}{article:blockchain-ethereum}
\addtocategory{articoli}{article:blockchain-explained}
\addtocategory{articoli}{article:blockchain-industries}
\addtocategory{articoli}{article:blockchain-misconceptions}
\addtocategory{articoli}{article:blockchain-revolution}
\addtocategory{articoli}{article:blockchain-school}
\addtocategory{articoli}{article:blockchain-works}

\addtocategory{articoli}{article:dart-async}
\addtocategory{articoli}{article:dart-introduction}

\addtocategory{articoli}{article:flutter-basics}
\addtocategory{articoli}{article:flutter-bloc}
\addtocategory{articoli}{article:flutter-future}
\addtocategory{articoli}{article:flutter-learning}
\addtocategory{articoli}{article:flutter-navigation}
\addtocategory{articoli}{article:flutter-review}
\addtocategory{articoli}{article:flutter-streams}

\addtocategory{manuali}{manual:devs-commercio-network}
\addtocategory{manuali}{manual:docs-commercio-network}
\addtocategory{manuali}{manual:sdk-commercio-network}

\addtocategory{web}{site:2c-solution}
\addtocategory{web}{site:android-studio}
\addtocategory{web}{site:commerc-io}
\addtocategory{web}{site:commercio-network}
\addtocategory{web}{site:dart}
\addtocategory{web}{site:flutter}
\addtocategory{web}{site:flutter-crash-course}
\addtocategory{web}{site:git}
\addtocategory{web}{site:github}
\addtocategory{web}{site:nginx}
\addtocategory{web}{site:powershell}
\addtocategory{web}{site:ubuntu}
\addtocategory{web}{site:virtual-box}

\defbibheading{articoli}{\section*{Riferimenti bibliografici}}
\defbibheading{manuali}{\section*{Manuali e documentazioni ufficiali}}
\defbibheading{web}{\section*{Siti web consultati}}

%**************************************************************
% Impostazioni di caption
%**************************************************************

\captionsetup{
    tableposition=top,
    figureposition=bottom,
    font=small,
    format=hang,
    labelfont=bf
}

%**************************************************************
% Impostazioni di glossaries
%**************************************************************

\makeglossaries
%**************************************************************
% Acronimi
%**************************************************************
\renewcommand{\acronymname}{Acronimi e abbreviazioni}

\newacronym[description=Ahead Of Time]
	{aot}{AOT}{Ahead Of Time}

\newacronym[description={\glslink{apig}{Application Program Interface}}]
    {api}{API}{Application Program Interface}
    
\newacronym[description=Commercio Cash Credit]
	{ccc}{CCC}{Commercio Cash Credit}
      
\newacronym[description=Commercio Token]
	{com}{COM}{Commercio Token}
    
\newacronym[description={\glslink{ddog}{DID Document}}]
	{ddo}{DDO}{DID Document}
    
\newacronym[description={\glslink{didg}{Decentralized Identifier}}]
    {did}{DID}{Decentralized Identifier}
    
\newacronym[description={\glslink{ictg}{Information and Communication Technologies}}]
	{ict}{ICT}{Information and Communication Technologies}
	
\newacronym[description=Just In Time]
	{jit}{JIT}{Just In Time}
	
\newacronym[description=Portable Document Format]
	{pdf}{PDF}{Portable Document Format}	

\newacronym[description={\glslink{pecg}{Posta Elettronica Certificata}}]
	{pec}{PEC}{Posta Elettronica Certificata}
	
\newacronym[description={\glslink{pocg}{Proof Of Concept}}]
	{poc}{POC}{Proof Of Concept}
	
\newacronym[description=Proof Of Stake]
	{pos}{POS}{Proof Of Stake}
	
\newacronym[description={\glslink{restg}{Representational State Transfer}}]
	{rest}{REST}{Representational State Transfer}
	
\newacronym[description={\glslink{sdkg}{Software Development Kit}}]
	{sdk}{SDK}{Software Development Kit}
	
\newacronym[description=Short Message Service]
	{sms}{SMS}{Short Message Service}

\newacronym[description={\glslink{spidg}{Sistema Pubblico d'Identità Digitale}}]
	{spid}{SPID}{Sistema Pubblico d'Identità Digitale}
	
\newacronym[description={\glslink{ssig}{Self-Sovereign Identity}}]
	{ssi}{SSI}{Self-Sovereign Identity}

\newacronym[description={\glslink{umlg}{Unified Modeling Language}}]
    {uml}{UML}{Unified Modeling Language}

%**************************************************************
% Glossario
%**************************************************************
\renewcommand{\glossaryname}{Glossario}

\newglossaryentry{apig}
{
    name=\glslink{api}{API},
    text=Application Program Interface,
    sort=api,
    description={in informatica con il termine \emph{Application Programming Interface (API)} (ing. interfaccia di programmazione di un'applicazione) si indica ogni insieme di procedure disponibili al programmatore, di solito raggruppate a formare una collezione di strumenti specifici per l'espletamento di un determinato compito all'interno di un certo programma. La finalità è ottenere un'astrazione, di solito tra l'hardware e il programmatore o tra software a basso e quello ad alto livello semplificando così il lavoro di programmazione}
}

\newglossaryentry{blockchaing}
{
	name=Blockchain,
	text=blockchain,
	sort=blockchain,
	description={in informatica con il termine \emph{blockchain} (ing. catena di blocchi) si indica una struttura dati condivisa e immutabile, le cui informazioni sono raggruppate in blocchi concatenati l'uno all'altro, a partire dal blocco d'origine, in ordine crescente per data di creazione, e la cui integrità viene garantita mediante l'uso della crittografia. Sebbene il numero di blocchi che la compongono sia destinato a crescere nel tempo, i dati contenuti al suo interno sono immutabili in quanto, una volta scritti non è più possibile né modificarli né eliminarli, a meno di non invalidare l'intera struttura. Il registro distribuito su cui la struttura si basa può essere acceduto, sia in lettura che in scrittura, da parte di tutti i nodi presenti nella rete che utilizza la \emph{blockchain} e, per garantire la coerenza tra le varie copie locali, l'aggiunta di un nuovo blocco viene gestita tramite un apposito protocollo di consenso condiviso}
}

\newglossaryentry{ddog}
{
	name=\glslink{ddo}{DDO},
	text=\glslink{did}{DID} Document,
	sort=ddo,
	description={in informatica con il termine \emph{\glslink{did}{DID} Document (DDO)} (ing. documento di un \glslink{did}{DID}) si indica un insieme di informazioni che descrivono il soggetto identificato da un \glslink{did}{DID}, corredati da opportuni meccanismi, come chiavi crittografiche e dati biometrici, attraverso i quali il soggetto può autenticare se stesso e dare prova della sua associazione al \glslink{did}{DID}}
}

\newglossaryentry{didg}
{
	name=\glslink{did}{DID},
	text=Decentralized Identifier,
	sort=did,
	description={in informatica con il termine \emph{Decentralized Identifier (DID)} (ing. identificativo decentralizzato) si indica un nuovo tipo di identificatore globale univoco che permette la creazione di identità digitali decentralizzate verificabili. Un \emph{DID} può identificare qualsiasi soggetto che necessiti di possedere un identificativo digitale e, a differenza delle altre identità digitali, non richiede la certificazione da parte di autorità centralizzate terze, fornitrici di attestazioni d'identità, per essere emesso; la registrazione tramite tecnologie \gls{blockchaing} o altre forme di reti decentralizzate, infatti, permette di risalire alle informazioni del soggetto identificato, senza bisogno di un'ente o di un'entità garante della loro veridicita e affidabilità}
}

\newglossaryentry{frameworkg}
{
	name=Framework,
	text=framework,
	sort=framework,
	description={in informatica con il termine \emph{framework} (ing. struttura o intelaiatura) si indica un'astrazione \textit{software} che fornisce diverse funzionalità generiche per lo sviluppo e la messa in produzione di applicazioni, permettendo comunque al programmatore di poterle modificare in base alle specifiche necessità del progetto. Rappresenta un ambiente \textit{software} universale e riutilizzabile che offre una piattaforma per facilitare lo sviluppo di applicazioni. Oltre a questo, può contenere al suo interno programmi di supporto, compilatori, librerire e \glslink{api}{API}}
}

\newglossaryentry{ictg}
{
	name=\glslink{ict}{ICT},
	text=Information and Communication Technologies,
	sort=ict,
	description={in informatica con il termine \emph{Information and Communication Technologies (ICT)} (ing. tecnologie di informazione e comunicazione) si indica ogni insieme di tecnologie riguardanti i sistemi integrati di telecomunicazione (linee di comunicazione cablate e senza fili), i \textit{computer}, le tecnologie audio-video e relativi \textit{software}, che permettono agli utenti di creare, immagazzinare e scambiare informazioni}
}

\newglossaryentry{namespaceg}
{
	name=Namespace,
	text=namespace,
	sort=namespace,
	description={in informatica con il termine \emph{namespace} (ing. spazio dei nomi) si indica un insieme di simboli, i nomi, che vengono utilizzati per identificare e riferire degli oggetti di vario genere. Lo scopo principale di un \emph{namespace} è assicurare che tutte le entità contenute al suo interno abbiano dei nomi univoci, in modo da poter essere facilmente identificabili. Gli spazi dei nomi sono spesso strutturati in gerarchie, in modo da poter riusare lo stesso nome per identificare oggetti distinti in contesti diversi.\\
	Nei linguaggi di programmazione, i \emph{namespace} sono tipicamente utilizzati per raggruppare un insieme di simboli e identificatori comuni ad una specifica funzionalità, in modo da evitare collisioni, ossia nomi duplicati che entrano in conflitto tra loro}
}

\newglossaryentry{nodo-validatoreg}
{
	name=Nodo validatore,
	text=nodo validatore,
	sort=nodo validatore,
	description={con il termine \emph{nodo validatore} si indica un nodo della rete Commercio.network che, oltre a mantenere in memoria tutto lo storico del \textit{database} distribuito che compone la \glslink{blockchaing}{blockchain}, ha anche la capacità di validare nuove transazioni che dovranno poi essere aggiunte in un nuovo blocco della catena. La validazione avviene tramite l'apposizione di una firma fatta con la propria chiave crittografica privata}
}

\newglossaryentry{open-sourceg}
{
	name=Open-source,
	text=open-source,
	sort=open-source,
	description={in informatica con il termine \emph{open-source} (ing. sorgente libero) si indica un \textit{software} di cui l'utente finale ha libero accesso a tutto il codice sorgente; questo può essere anche modificato per cambiarne e/o evolverne il funzionamento oppure per correggerne eventuali errori riscontrati durante l'utilizzo. L'utente inoltre ha anche la possibilità di ridistribuire pubblicamente la versione da lui rielaborata, infatti, rilasciando un \textit{software} con una licenza \emph{open-source}, gli sviluppatori rinunciano a qualsiasi diritto di proprietà intelluttuale sullo stesso}
}

\newglossaryentry{pecg}
{
	name=\glslink{pec}{PEC},
	text=Posta Elettronica Certificata,
	sort=pec,
	description={in informatica con il termine \emph{Posta Elettronica Certificata (PEC)} si indica il sistema di posta elettronica che fornisce al mittente ricevuta con valore legale dell’invio e della ricezione di messaggi informatici. La \textit{PEC} è quindi un sistema in grado di certificare l'identità (digitale) di un soggetto che ne fa utilizzo, garantendo inoltre il non ripudio per tutte le operazioni da esso svolte}
}

\newglossaryentry{pocg}
{
	name=\glslink{poc}{POC},
	text=proof of concept,
	sort=poc,
	description={in informatica con il termine \emph{Proof Of Concept (POC)} (ing. prova di concetto) si indica la realizzazione di un \textit{software} allo scopo di dimostrare la fattibilità di un certo principio o di una certa idea che sta alla base di un progetto \textit{software} più ampio e completo. Un \emph{Proof Of Concept (POC)} è solitamente di dimensioni ridotte e può anche essere incompleto, questo perché il suo fine è solamente verificare il potenziale pratico di un concetto teorico}
}

\newglossaryentry{power-upg}
{
	name=Powerup,
	text=powerup,
	sort=powerup,
	description={nel contesto della Commercio.network, con il termine \emph{powerup} (ing. potenziamento), si indica l'operazione attraverso cui un nodo della rete, identificato da un proprio \glslink{did}{DID}, richiede il trasferimento di un certo ammontare di \glslink{tokeng}{token} dal proprio \glslink{walletg}{wallet} verso quello di un altro nodo, identificato a sua volta da un suo \glslink{did}{DID}}
}

\newglossaryentry{restg}
{
	name=\glslink{rest}{REST},
	text=Representational State Transfer,
	sort=rest,
	description={in informatica con il termine \emph{Representational State Transfer (REST)} (ing. trasferimento di stato rappresentativo) si indica uno stile architetturale \textit{software} che definisce un insieme di vincoli da rispettare nella creazione di servizi \textit{web}. I servizi \textit{web} conformi a questi principi vengono chiamati \textit{RESTful} e forniscono un alto grado di interoperabilità tra i diversi sistemi che comunicano in rete. Infatti permettono di accedere e di elaborare una rappresentazione testuale delle risorse esposte dai servizi stessi, utilizzando un insieme di operazioni uniformi e predefinite.\\
	Ogni risorsa, identificata da uno specifico nome univoco, può essere acceduta sia in lettura che in scrittura, in base ai permessi previsti per la stessa, attraverso appositi metodi del protocollo utilizzato per la gestione delle richieste da parte dei \textit{client} e le risposte da parte del \textit{server}}
}

\newglossaryentry{sdkg}
{
	name=\glslink{sdk}{SDK},
	text=Software Development Kit,
	sort=sdk,
	description={in informatica con il termine \emph{Software Development Kit (SDK)} (ing. pacchetto di sviluppo per \textit{software}) si indica una collezione di strumenti per lo sviluppo \textit{software} contenuti all'interno di un singolo pacchetto installabile all'interno del proprio sistema; questi permettono di facilitare, a volte notevolmente, la creazione di applicazioni. Questi strumenti solitamente sono specifici per il particolare tipo di \textit{hardware}, sistema operativo e linguaggio di programmazione utilizzati per lo sviluppo \textit{software}}
}

\newglossaryentry{spidg}
{
	name=\glslink{spid}{SPID},
	text=Sistema Pubblico d'Identità Digitale,
	sort=spid,
	description={con il termine \emph{Sistema Pubblico d'Identità Digitale (SPID)} si indica la piattaforma tramite cui le pubbliche amministrazioni consentono ai cittadini e alle imprese l'accesso in rete ai propri servizi tramite un unico sistema di \textit{password}. Questo sistema consente quindi di identitificare univocamente ogni soggetto che ne fa utilizzo, fornendogli un'identità digitale che lo rappresenta e che gli permette di usufruire di tutti quei servizi che richiedono un'autenticazione tramite documento d'identificazione}
}

\newglossaryentry{ssig}
{
	name=\glslink{ssi}{SSI},
	text=Self-Sovereign Identity,
	sort=ssi,
	description={in informatica con il termine \emph{Self-Sovereign Identity (SSI)} (ing. identità sovrana) si indica un meccanismo che permette agli individui e agli enti di creare la propria identità digitale, mantenendo il pieno controllo delle realtive credenziali, senza bisogno di richiederne l'emissione certificata da parte di un intermediaro terzo o di un'autorità centralizzata. I titolari delle identità hanno la totale padronanza dei propri dati personali e possono decidere in autonomia in che termini questi possono essere utilizzati e condivisi}
}

\newglossaryentry{tokeng}
{
	name=Token,
	text=token,
	sort=token,
	description={in informatica, all'interno del contesto di una \glslink{blockchaing}{blockchain}, con il termine \emph{token} (ing. gettone) si indica un bene digitale che può essere visto come una collezione di informazioni in grado di garantire un diritto di proprietà, al soggetto possessore, su un certo insieme di dati presenti all'interno della \glslink{blockchaing}{blockchain} stessa. I \emph{token} possono essere scambiati tra due diversi partecipanti alla rete senza bisogno di un intermediario, ma semplicemente effettuando una transazione; infatti questi non sono emessi da un ente, ma fanno nativamente parte della \glslink{blockchaing}{blockchain} in cui vengono generati.\\
	Oltre a questo, possono essere utilizzati come delle vere e proprie monete digitali: in questo caso assumono il nome di criptovalute in quanto la loro progettazione si basa in gran parte su tecnologie crittografiche. Il loro valore viene dettato dalle leggi economiche del mercato, in particolare da quella della domanda e dell'offerta, applicata ai nodi della rete; infatti il loro controvalore in valuta legale è spesso molto variabile}
}

\newglossaryentry{umlg}
{
    name=\glslink{uml}{UML},
    text=UML,
    sort=uml,
    description={in ingegneria del \textit{software} l'\emph{Unified Modeling Language (UML)} (ing. linguaggio di modellazione unificato) è un linguaggio di modellazione e specifica basato sul paradigma \textit{object-oriented}. L'\emph{UML} svolge un'importantissima funzione di ``lingua franca'' nella comunità della progettazione e programmazione a oggetti. Gran parte della letteratura di settore usa tale linguaggio per descrivere soluzioni analitiche e progettuali in modo sintetico e comprensibile a un vasto pubblico}
}

\newglossaryentry{walletg}
{
	name=Wallet,
	text=wallet,
	sort=wallet,
	description={in informatica, all'interno del contesto di una \glslink{blockchaing}{blockchain}, con il termine \emph{wallet} (ing. portafoglio) si indica un sistema \textit{hardware} o \textit{software} che permette di tracciare l'ammontare del numero di \glslink{tokeng}{token}, o criptovalute, possedute da un utente della \glslink{blockchaing}{blockchain}. In realtà il \emph{wallet} non contiene effettivamente al suo interno le criptovalute, queste sono invece memorizzate dentro al \textit{database} distribuito che costituisce la \glslink{blockchaing}{blockchain} stessa; infatti, attraverso il \emph{wallet} è possibile solamente scambiare (inviare o ricevere) dei \glslink{tokeng}{token}, con gli altri utenti partecipanti alla rete, tramite delle transazioni. Per fare questo viene mantenuta una coppia di chiavi crittografiche, una privata ed una pubblica, all'interno del \emph{wallet} stesso: la chiave pubblica serve per ricevere dei versamenti, mentre la chiave privata viene utilizzata per effettuare dei pagamenti dal proprio \emph{wallet}}
}
 										% database di termini

%**************************************************************
% Impostazioni di graphicx
%**************************************************************

\graphicspath{{immagini/}}								% cartella dove sono riposte le immagini

%**************************************************************
% Impostazioni di hyperref
%**************************************************************

\hypersetup{
    %hyperfootnotes=false,
    %pdfpagelabels,
    %draft,	% = elimina tutti i link (utile per stampe in bianco e nero)
    colorlinks=true,
    linktocpage=true,
    pdfstartpage=1,
    pdfstartview=FitV,
    % decommenta la riga seguente per avere link in nero (per esempio per la stampa in bianco e nero)
    %colorlinks=false, linktocpage=false, pdfborder={0 0 0}, pdfstartpage=1, pdfstartview=FitV,
    breaklinks=true,
    pdfpagemode=UseNone,
    pageanchor=true,
    pdfpagemode=UseOutlines,
    plainpages=false,
    bookmarksnumbered,
    bookmarksopen=true,
    bookmarksopenlevel=1,
    hypertexnames=true,
    pdfhighlight=/O,
    %nesting=true,
    %frenchlinks,
    urlcolor=webbrown,
    linkcolor=RoyalBlue,
    citecolor=webgreen,
    %pagecolor=RoyalBlue,
    %urlcolor=Black, linkcolor=Black, citecolor=Black,
    %pagecolor=Black,
    pdftitle={\myTitle},
    pdfauthor={\textcopyright\ \myName, \myUni, \myFaculty},
    pdfsubject={},
    pdfkeywords={},
    pdfcreator={pdfLaTeX},
    pdfproducer={LaTeX}
}

%**************************************************************
% Impostazioni di itemize
%**************************************************************

%\renewcommand{\labelitemi}{$\ast$}

%\renewcommand{\labelitemi}{$\bullet$}
\renewcommand{\labelitemii}{$\cdot$}
%\renewcommand{\labelitemiii}{$\diamond$}
%\renewcommand{\labelitemiv}{$\ast$}

%**************************************************************
% Impostazioni di listings
%**************************************************************

\lstset{
    language=[LaTeX]Tex,								%C++,
    keywordstyle=\color{RoyalBlue}, 					%\bfseries,
    basicstyle=\small\ttfamily,
    %identifierstyle=\color{NavyBlue},
    commentstyle=\color{Green}\ttfamily,
    stringstyle=\rmfamily,
    numbers=none, 										%left,
    numberstyle=\scriptsize, 							%\tiny
    stepnumber=5,
    numbersep=8pt,
    showstringspaces=false,
    breaklines=true,
    frameround=ftff,
    frame=single
} 

%**************************************************************
% Impostazioni di xcolor
%**************************************************************

\definecolor{webgreen}{rgb}{0,.5,0}
\definecolor{webbrown}{rgb}{.6,0,0}
\definecolor{grayer}{HTML}{cccccc} % funzionali

%**************************************************************
% Altro
%**************************************************************

\newcommand{\omissis}{[\dots\negthinspace]} 			% produce [...]

% eccezioni all'algoritmo di sillabazione
\hyphenation
{
    ma-cro-istru-zio-ne
    gi-ral-din
}

\newcommand{\sectionname}{sezione}
\addto\captionsitalian{\renewcommand{\figurename}{Figura}
                       \renewcommand{\tablename}{Tabella}}

\newcommand{\glsfirstoccur}{\ap{{[g]}}}

\newcommand{\intro}[1]{\emph{\textsf{#1}}}

%**************************************************************
% Environment per ``rischi''
%**************************************************************

\newcounter{riskcounter}               					% define a counter
\setcounter{riskcounter}{0}             				% set the counter to some initial value

%%%% Parameters
% #1: Title
\newenvironment{risk}[1]{
    \refstepcounter{riskcounter}        				% increment counter
    \par \noindent                     	 				% start new paragraph
    \textbf{\arabic{riskcounter}. #1}   				% display the title before the 
                                        				% content of the environment is displayed 
}{
    \par\medskip
}

\newcommand{\riskname}{Rischio}

\newcommand{\riskdescription}[1]{\textbf{\\Descrizione:} #1.}

\newcommand{\risksolution}[1]{\textbf{\\Soluzione:} #1.}

%**************************************************************
% Environment per ``use case''
%**************************************************************

\newcounter{usecasecounter}             				% define a counter
\setcounter{usecasecounter}{0}          				% set the counter to some initial value

%%%% Parameters
% #1: ID
% #2: Nome
\newenvironment{usecase}[2]{
    \renewcommand{\theusecasecounter}{\usecasename #1}  % this is where the display of 
                                                        % the counter is overwritten/modified
    \refstepcounter{usecasecounter}             		% increment counter
    \vspace{10pt}
    \par \noindent                              		% start new paragraph
    {\large \textbf{\usecasename #1: #2}}       		% display the title before the 
                                               			% content of the environment is displayed 
    \medskip
}{
    \medskip
}

\newcommand{\usecasename}{UC}

\newcommand{\usecaseactors}[1]{\textbf{\\Attori Principali:} #1. \vspace{4pt}}
\newcommand{\usecasepre}[1]{\textbf{\\Precondizioni:} #1. \vspace{4pt}}
\newcommand{\usecasedesc}[1]{\textbf{\\Descrizione:} #1. \vspace{4pt}}
\newcommand{\usecasepost}[1]{\textbf{\\Postcondizioni:} #1. \vspace{4pt}}
\newcommand{\usecasealt}[1]{\textbf{\\Scenario Alternativo:} #1. \vspace{4pt}}

%**************************************************************
% Environment per ``namespace description''
%**************************************************************

\newenvironment{namespacedesc}{
    \vspace{10pt}
    \par \noindent                             			% start new paragraph
    \begin{description} 
}{
    \end{description}
    \medskip
}

\newcommand{\classdesc}[2]{\item[\textbf{#1:}] #2}
