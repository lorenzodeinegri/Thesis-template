% !TEX encoding = UTF-8
% !TEX TS-program = pdflatex
% !TEX root = ../tesi.tex

%**************************************************************
% Sommario
%**************************************************************
\cleardoublepage
\phantomsection
\pdfbookmark{Sommario}{Sommario}
\begingroup
\let\clearpage\relax
\let\cleardoublepage\relax
\let\cleardoublepage\relax

\chapter*{Sommario}

Il presente documento descrive il lavoro svolto durante il periodo di stage, della durata di trecento ore, dal laureando \myName{} presso l'azienda \myCompany{} \companyTitle{}, situata a Limena (PD).\\
Lo stage ha avuto come argomento principale la \textit{blockchain}, le tecnologie ad essa correlate e come queste possono essere utilizzate per la creazione di identità digitali e la condivisione di documenti firmati. Il progetto, nella sua interezza, è di particolare rilevanza per l'azienda, in quanto direttamente coinvolto nel raggiungimento della sua \textit{mission}: eliminare in maniera definitiva la generazione di documenti cartacei all'interno delle aziende.
Gli obiettivi da raggiungere erano molteplici e sono stati divisi in due parti distinte, una teorica di ricerca e formazione sull'argomento trattato e una pratica di progettazione e sviluppo di un relativo \textit{proof of concept}.\\
Nel primo periodo, dopo la necessaria formazione sul contesto aziendale, era richiesto lo studio teorico di tutti i concetti legati alla \textit{blockchain}, analizzandone il funzionamento, lo scopo, i possibili utilizzi in casi reali, i pregi e i difetti; successivamente bisognava analizzare, allo stesso modo, la \textit{blockchain} sviluppata da Commerc.io S.r.l., una start-up innovativa associata a \myCompany{} che punta alla realizzazione di una rete di aziende che consenta la firma e lo scambio di documenti tra i partecipanti. Infine, in preparazione alla realizzazione del \textit{PoC} previsto dal progetto, era necessario approfondire la documentazione della \textit{blockchain} commercio.network dal punto di vista degli strumenti di sviluppo, quali librerie e \textit{SDK}, disponibili per interagire con la rete stessa.\\
Nel secondo periodo era richiesta l'analisi, la progettazione e l'implementazione di un'applicazione \textit{mobile}, scritta  in linguaggio Dart attraverso il \textit{framework} Flutter, per la creazione di identità digitali e la condivisione di documenti tramite la rete commercio.network. Inoltre, per raggiungere tale obiettivo, è stato necessario un periodo di formazione individuale sugli strumenti di sviluppo scelti dall'azienda per la realizzazione dell'applicazione.

%\vfill
%
%\selectlanguage{english}
%\pdfbookmark{Abstract}{Abstract}
%\chapter*{Abstract}
%
%\selectlanguage{italian}

\endgroup			

\vfill

