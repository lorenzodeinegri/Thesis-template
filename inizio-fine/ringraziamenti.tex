% !TEX encoding = UTF-8
% !TEX TS-program = pdflatex
% !TEX root = ../tesi.tex

%**************************************************************
% Ringraziamenti
%**************************************************************
\cleardoublepage
\phantomsection
\pdfbookmark{Ringraziamenti}{ringraziamenti}

\begin{flushright}{
	\slshape    
	``Anyone who has lost track of time when using a computer knows the propensity to dream, the urge to make dreams come true and the tendency to miss lunch.''} \\ 
	\medskip
	--- Tim Berners-Lee
\end{flushright}

\bigskip

\begingroup
\let\clearpage\relax
\let\cleardoublepage\relax
\let\cleardoublepage\relax

\chapter*{Ringraziamenti}

\noindent \textit{Innanzitutto, vorrei esprimere la mia gratitudine al \profTitle{} \myProf, relatore della mia tesi, sia per l'aiuto e il sostegno fornitomi durante la stesura del lavoro e nel corso del progetto di stage, sia per la passione trasmessa nello studio di alcuni temi dell'informatica. La dedizione nei confronti del suo lavoro è stata per me stimolo ed ispirazione.}\\

\noindent \textit{Desidero ringraziare con affetto la mia famiglia per il supporto, anche economico, e il grande aiuto che mi hanno dato durante tutto il periodo di studi. La stabilità, che non mi hanno mai fatto mancare, mi ha permesso di affrontare il percorso universitario con la serenità e la concentrazione necessaria.}\\

\noindent \textit{Voglio dedicare un pensiero speciale a Fiammetta per lo splendido rapporto che abbiamo costruito durante questi anni e per essermi stata sempre vicina, sostenendomi nei momenti difficili e condividendo insieme quelli più felici.}\\

\noindent \textit{Infine, desidero ringraziare di cuore tutti i miei amici per i bellissimi momenti passati insieme durante questi tre anni e le mille avventure vissute. Un ringraziamento particolare va a Massimo per essere sempre stato pronto ad aiutarmi nel momento del bisogno, sia nella sfera universitaria che in quella personale, e per aver superato insieme ogni fatica ed ogni ostacolo che abbiamo trovato durante il nostro percorso di studi.}\\
\bigskip

\noindent\textit{\myLocation, \myTime}
\hfill \myName

\endgroup

