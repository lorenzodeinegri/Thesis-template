% !TEX encoding = UTF-8
% !TEX TS-program = pdflatex
% !TEX root = ../tesi.tex

%**************************************************************
\chapter{Conclusioni}
\label{cap:conclusioni}

%**************************************************************
\section{Consuntivo finale}

Al termine del progetto di stage, le ore di lavoro effettivamente svolte sono risultate in linea con quanto previsto all'interno del piano di lavoro e ammontano ad un totale di 300.\\
Pur rispettando i tempi inizialmente preventivati, si sono verificati degli scostamenti orari, sia in positivo che in negativo, a livello delle singole settimane. In particolare, sebbene sia stato impiegato più tempo del previsto per la formazione iniziale, in quanto il contesto aziendale era completamente nuovo e sconosciuto, la fase di studio e ricerca sulle tecnologie da utilizzare nelle settimane successive è stata svolta più rapidamente, grazie al buon materiale documentale a disposizione. Successivamente, l'implementazione delle funzionalità di base dell'applicazione ha registrato delle discrepanze orarie minime, principalmente dovute al presentarsi o meno di \textit{bug} durante lo sviluppo. La differenza maggiore è stata riscontrata nell'ultimo periodo di realizzazione del \gls{poc}, tuttavia questa non è stata causata da delle problematiche impreviste, bensì dall'investimento di tutto il tempo che era stato risparmiato in precedenza, in modo da riuscire a soddisfare più requisiti possibile dell'applicazione. Le ultime settimane, invece, sono rimaste in linea con quanto preventivato e tutto il lavoro è stato concluso entro i termini previsti.\\
Di seguito viene mostrata la tabella riportante il consuntivo finale del progetto di stage.

\rowcolors{1}{grayer}{white}
\begin{longtable}{|p{5.6cm}|c|c|c|}
	\hline
	\rowcolor{gray}
	\textbf{Attività} & \textbf{Preventivo} & \textbf{Consuntivo} & \textbf{Differenza} \\\hline
	Studio dei concetti fondamentali legati alla \gls{blockchaing} e del contesto aziendale. & 20 & 24 & +4 \\\hline
	Studio approfondito della \gls{blockchaing} di Commerc.io S.r.l. e delle funzionalità da implementare. & 40 & 32 & -8 \\\hline
	Implementazione e verifica delle funzionalità offerte ad un \textit{account} registrato. & 40 & 36 & -4 \\\hline
	Implementazione e verifica delle funzionalità di gestione dell'identità digitale. & 40 & 36 & -4 \\\hline
	Implementazione e verifica delle funzionalità di condivisione di documenti. & 40 & 44 & +4 \\\hline
	Implementazione dell'interfaccia grafica e verifica del prototipo. & 40 & 48 & +8 \\\hline
	Stesura della documentazione ed inizio della relazione. & 40 & 40 & 0 \\\hline
	Termine stesura della relazione e presentazione di quanto svolto in azienda. & 40 & 40 & 0 \\\hline
	
	\caption{Tabella del consuntivo della pianificazione dello stage}
	\label{tab:rendiconto}
\end{longtable}

%**************************************************************
\section{Raggiungimento degli obiettivi}

Al termine del progetto di stage sono stati raggiunti tutti gli obiettivi obbligatori e l'obiettivo desiderabile; quello facoltativo non è stato trattato in quanto, dopo un'attenta valutazione dei tempi necessari per affrontarlo, in rapporto con quelli previsti dal piano di lavoro, è stato deciso di dedicare più lavoro possibile per la realizzazione dell'obiettivo desiderabile, eventualmente andando oltre quanto richiesto inizialmente per lo stesso.

\subsection{Obbligatori}

\begin{itemize}
	\item \textbf{RO-1 raggiunto:} è stato svolto un lavoro di ricerca e di studio riguardo le tematiche relative alla \gls{blockchaing} e alle tecnologie associate, approfondendo poi quelle che sarebbero state utilizzate durante il successivo sviluppo dell'applicazione;
	\item \textbf{RO-2 raggiunto:} è stato implementato un \gls{poc} che interagisce con la \gls{blockchaing} di Commerc.io S.r.l. per la gestione delle identità digitali e la condivisione di documenti firmati;
	\item \textbf{RO-3 raggiunto:} è stata stesa la documentazione relativa a tutti i risultati ottenuti nel corso delle varie fasi del progetto, con particolare riferimento alla gestione e risoluzione delle difficoltà incontrate;
	\item \textbf{RO-4 raggiunto:} è stata stesa una relazione riguardante il lavoro svolto durante tutto il periodo di stage, mettendo in luce le tecnologie e gli strumenti utilizzati, oltre alle procedure di installazione, configurazione ed utilizzo.
\end{itemize}

\subsection{Desiderabili}

\begin{itemize}
	\item \textbf{RD-1 raggiunto:} è stata sviluppata un'applicazione per dispositivi mobili, tramite il \gls{frameworkg} Flutter, che offre un'interfaccia grafica in grado di fornire l'accesso alle funzionalità implementate con il \gls{poc}.
\end{itemize}

\subsection{Facoltativi}

\begin{itemize}
	\item \textbf{RF-1 non raggiunto:} è stato deciso di non iniziare l'implementazione di un'interfaccia \textit{web} per il \gls{poc}, preferendo dedicare tutto il tempo a disposizione all'applicazione per dispositivi mobili.
\end{itemize}

\subsection{Riepilogo}

Di seguito viene mostrata la tabella riportante il resoconto di tutti gli obiettivi con il relativo stato alla data di fine stage.

\rowcolors{1}{grayer}{white}
\begin{longtable}{|c|p{7.8cm}|c|}
	\hline
	\rowcolor{gray}
	\textbf{Obiettivo} & \textbf{Descrizione} & \textbf{Stato} \\\hline
	RO-1 & Acquisizione delle competenze sulle tematiche relative alla \gls{blockchaing} e alle tecnologie associate. & Raggiunto \\\hline
	RO-2 & Implementazione di un \gls{poc} che permetta l'interazione con la \gls{blockchaing} di Commerc.io S.r.l. per il riconoscimento delle identità digitali e la trasmissione di documenti, tramite un'interfaccia grafica basilare. & Raggiunto \\\hline
	RO-3 & Stesura della documentazione relativa ai risultati ottenuti nel corso delle varie fasi del progetto e alla gestione delle difficoltà incontrate. & Raggiunto \\\hline
	RO-4 & Stesura della relazione riguardante il lavoro svolto durante tutto il periodo di stage. & Raggiunto \\\hline
	RD-1 & Sviluppo di un'applicazione \textit{mobile} in Flutter che offra le funzionalità richieste per il \gls{poc}. & Raggiunto \\\hline
	RF-1 & Implementazione di un'interfaccia \textit{web} completa per il \gls{poc}. & Non raggiunto \\\hline
	
	\caption{Tabella dello stato degli obiettivi dello stage}
	\label{tab:raggiungimento-obiettivi}
\end{longtable}

%**************************************************************
\section{Consegna dei prodotti}

Al termine del progetto di stage sono stati consegnati tutti i prodotti obbligatori e il prodotto desiderabile; quello facoltativo non è stato trattato in quanto, dopo un'attenta valutazione dei tempi necessari per affrontarlo, in rapporto con quelli previsti dal piano di lavoro, è stato deciso di dedicare più lavoro possibile per la realizzazione del prodotto desiderabile, eventualmente andando oltre quanto richiesto inizialmente per lo stesso.

\subsection{Obbligatori}
\begin{itemize}
	\item è stato consegnato il \gls{poc} che mostra il funzionamento della gestione delle identità digitali e del trasferimento di documenti firmati tramite la \gls{blockchaing} \textit{Commercio.network}, sviluppata da Commerc.io S.r.l.;
	\item è stata consegnata la documentazione dettagliata dei risultati ottenuti nel corso delle varie fasi del progetto, con un'analisi critica di tutte le difficoltà incontrate e le relative soluzione adottate;
	\item è stata consegnata una relazione riguardante il lavoro svolto durante tutto il periodo di stage, mettendo in luce le tecnologie e gli strumenti utilizzati, oltre alle procedure di installazione, configurazione ed utilizzo;
	\item è stata consegnata una presentazione che espone il lavoro svolto e le conoscenze apprese, le tecnologie e gli strumenti utilizzati, il prodotto realizzato e il suo funzionamento.
\end{itemize}

\subsection{Desiderabili}

\begin{itemize}
	\item è stata consegnata un'applicazione per dispositivi mobili, realizzata tramite il \gls{frameworkg} Flutter, che offre un'interfaccia grafica in grado di fornire l'accesso alle funzionalità implementate con il \gls{poc}.
\end{itemize}

\subsection{Facoltativi}

\begin{itemize}
	\item non è stata consegnata un'interfaccia \textit{web} per il \gls{poc}, in quanto la sua implementazione è stata esclusa dal progetto di stage, preferendo dedicare tutto il tempo a disposizione all'applicazione per dispositivi mobili.
\end{itemize} 

\subsection{Riepilogo}

Di seguito viene mostrata la tabella riportante il resoconto di tutti i prodotti con il relativo stato della consegna alla data di fine stage.

\rowcolors{1}{grayer}{white}
\begin{longtable}{|p{9.7cm}|c|}
	\hline
	\rowcolor{gray}
	\textbf{Prodotto} & \textbf{Stato} \\\hline
	Il \gls{poc} che mostri il funzionamento della gestione delle identità digitali e del trasferimento di documenti tramite \gls{blockchaing}, che faccia uso delle \gls{api} sviluppate da Commerc.io S.r.l. e offra una semplice interfaccia per il suo utilizzo. & Consegnato \\\hline
	La documentazione dettagliata dei risultati ottenuti, comprendente un'analisi critica delle difficoltà riscontrate e delle soluzioni adottate. & Consegnato \\\hline
	Una relazione che illustri le attività svolte e i prodotti realizzati. & Consegnato \\\hline
	Una presentazione finale del lavoro svolto, da esporre in azienda. & Consegnato \\\hline
	Un'applicazione \textit{mobile} scritta in Flutter che offra le funzionalità richieste dal progetto. & Consegnato \\\hline
	Un'interfaccia \textit{web} completa che permetta l'utilizzo del \gls{poc}. & Non consegnato \\\hline
	
	\caption{Tabella dello stato della consegna dei prodotti dello stage}
	\label{tab:consegna-prodotti}
\end{longtable}

%**************************************************************
\section{Analisi retrospettiva}

Al termine del progetto di stage è stata effettuata un'analisi critica di tutto il lavoro svolto durante i due mesi che hanno costituito il periodo di tirocinio aziendale. In questo modo è stato possibile prendere consapevolezza del bagaglio formativo acquisito e dell'importanza che questo può avere per le esperienze lavorative future, non solo dal punto di vista tecnico professionale, ma anche da quello personale. 

\subsection{Contenuti formativi}

Durante tutte le attività svolte nel corso dello stage, sono stati affrontati diversi argomenti non presenti all'interno del piano di studi offerto dalla Laurea Triennale in Informatica, in quanto relativi a tematiche innovative e/o molto specifiche. Tuttavia, grazie alle nozioni di base fornite nel corso dei tre anni di studi universitari, è stato possibile svolgere il periodo di formazione e studio iniziale in maniera sostenibile, sia per il tempo che per le risorse impiegate.    

\subsubsection*{Conoscenze}

Le attività lavorative svolte hanno portato all'apprendimento di nuove conoscenze tecniche e al raffinamento di quelle già padroneggiate. Tra queste le più importanti sono:

\begin{itemize}
	\item \textbf{\gls{blockchaing}:} è stato possibile trattare le tecnologie \gls{blockchaing} prima da un punto di vista teorico, studiandone la struttura, il funzionamento e le caratteristiche specifiche, e poi da un punto di vista più pratico, analizzando una \gls{blockchaing} realmente utilizzata in un contesto produttivo, in modo da poter apprezzare i pregi e i difetti che il suo uso comporta e quindi sviluppare un certo grado di consapevolezza riguardo i possibili ambiti applicativi dove può essere utile il suo impiego;
	\item \textbf{linguaggio di programmazione Dart:} il \gls{poc} di cui è stato previsto lo sviluppo è stato scritto utilizzando il linguaggio di programmazione Dart, di conseguenza è stato prima necessario compiere uno studio delle regole sintattiche e delle funzionalità che fornisce agli sviluppatori, per poi applicare quanto appreso nell'implementazione del \textit{software} richiesto;
	\item \textbf{\gls{frameworkg} di sviluppo Flutter:} l'applicazione per dispositivi mobili, avente lo scopo di fornire un'interfaccia grafica utile ad esporre le funzioni implementate con il \gls{poc}, è stata realizzata tramite il \gls{frameworkg} Flutter, quindi è stato prima necessario capirne il funzionamento e i casi d'uso, per poi sfruttarne le funzionalità e gli strumenti di sviluppo messi a disposizione, per creare l'applicazione richiesta.
\end{itemize}

\subsubsection*{Competenze}

Il progetto di stage svolto insieme all'azienda ha indotto un percorso di maturazione e crescita professionale che ha portato all'acquisizione di nuove competenze personali. Tra queste le più significative sono:

\begin{itemize}
	\item \textbf{formazione individuale autonoma:} tutto il primo periodo del progetto di stage prevedeva un lavoro di formazione individuale autonoma su tecnologie e strumenti innovativi e, talvolta, non ancora completamente maturi, allo scopo di conoscerne le caratteristiche, il funzionamento e di comprendere i pregi e i difetti di una loro possibile applicazione pratica; è stato quindi necessario provare diverse strategie di ricerca, in modo da trovare quella migliore per efficacia ed efficienza;
	\item \textbf{sviluppo di applicazioni per dispositivi mobili:} il secondo periodo del progetto di stage richiedeva lo sviluppo di un'applicazione per dispositivi mobili che fornisse tutte le funzionalità previste per il \gls{poc}, di conseguenza, è stato prima necessario approfondire il funzionamento di tali \textit{software} e le specifiche problematiche che devono essere risolte durante il loro sviluppo, per poi procedere con l'implementazione;
	\item \textbf{comunicazioni e relazioni interpersonali}: durante tutto il progetto di stage è stato necessario mantenere una comunicazione costante con il tutor aziendale e, occasionalmente, anche con altri colleghi per prendere decisioni di particolare rilevanza per lo sviluppo del \gls{poc} oppure per risolvere specifiche problematiche riscontrate; si è rivelato fondamentale trovare una modalità di comunicazione chiara, che non lasciasse spazio ad equivoci ed incomprensioni, in modo da evitare inefficienze difficilmente accettabili in un contesto lavorativo.
\end{itemize}

\subsubsection*{Tecnologie e strumenti}

Nelle attività svolte durante il progetto di stage, è stato richiesto l'uso di molte tecnologie nuove e differenti, in base alle esigenze e alle richieste da parte aziendali, portando ad un aumento del numero di strumenti padroneggiati e delle abilità nell'utilizzo di quelli già conosciuti. Tra questi i principali sono:

\begin{itemize}
	\item \textbf{Macchine virtuali e VirtualBox:} è stata utilizzata una macchina virtuale, gestita tramite VirtualBox, per la predisposizione dell'ambiente di esecuzione del nodo della \gls{blockchaing} usato nel progetto; entrambi sono stati già incontrati durante gli studi universitari, quindi non è stato necessario affrontare alcuna fase di apprendimento prima del loro impiego;
	\item \textbf{Ubuntu e \textit{Commercio.network}:} è stato utilizzato Ubuntu per installare i \textit{software} necessari all'esecuzione del nodo della rete \textit{Commercio.network} e quelli per renderlo accessibile dall'esterno, in modo da poter poi interagire con la \gls{blockchaing}; sebbene il sistema operativo fosse già stato usato nel percorso di studi, i \textit{software} sviluppati da Commerc.io S.r.l. sono proprietari, tuttavia è stata fornita una documentazione per l'installazione molto dettagliata che ha resto la procedura semplice e lineare;
	\item \textbf{Git e GitHub:} sono stati utilizzati Git e GitHub per gestire il versionamento del codice del \gls{poc} sviluppato, mantenere lo storico di tutte le modifiche effettuate e condividere il progetto \textit{software} con il tutor aziendale; entrambi gli strumenti sono stati già utilizzati durante il corso di Tecnologie Open-Source, quindi il loro funzionamento era già ampiamente conosciuto;
	\item \textbf{Dart, Flutter e Android Studio:} sono stati utilizzati il linguaggio di programmazione Dart e il \gls{frameworkg} Flutter per sviluppare tutta l'applicazione per dispositivi mobili, attraverso l'ambiente di sviluppo integrato Android Studio; sebbene quest'ultimo fosse già conosciuto, in quanto molto simile ad altri ambienti di sviluppo utilizzati durante i progetti universitari, sia il linguaggio che il \gls{frameworkg} non erano mai stati usati e hanno quindi richiesto un periodo di formazione individuale prima di poter procedere con lo sviluppo.
\end{itemize}

\subsection{Metodo di lavoro}

A causa del distanziamento sociale imposto per prevenire il diffondersi del virus Covid-19, lo stage è stato svolto per la maggior parte del tempo da remoto, tramite \textit{smart working}, perciò non è stato sempre possibile avere un'interazione diretta con il tutor aziendale. Tuttavia, nonostante la novità e il poco tempo per organizzare le attività al meglio, sfruttando appositi strumenti di comunicazione telematica predisposti dall'azienda, quali Skype e Microsoft Teams, veniva fissato un incontro al termine di ogni giornata lavorativa, al fine di verificare i progressi ottenuti. In questo modo è stato possibile monitorare costantemente lo stato di avanzamento del progetto e quindi garantire il raggiungimento degli obiettivi fissati nella pianificazione, risolvendo prontamente eventuali problematiche riscontrate.\\
In ogni caso, è sempre stato possibile comunicare con il tutor aziendale tramite \textit{email}, \textit{chat} testuali o \textit{chat} vocali in caso di necessità, senza una preventiva pianificazione. Inoltre, per tracciare tutte le attività svolte durante lo stage è stato ritenuto fosse il caso di riportare un riassunto giornaliero chiaro e coinciso delle attività stesse e delle eventuali problematiche riscontrate, all’interno di un apposito documento condiviso con il tutor.\\
Grazie a tutti questi accorgimenti è stato possibile lavorare in modo efficiente ed efficace anche dalla propria abitazione, senza inficiare la qualità dal lavoro svolto e consentendo il raggiungimento di tutti gli obiettivi pianificati, mantenendo al contempo tutti i vantaggi dello \textit{smart working}, quali la flessibilità oraria e il risparmio dei tempi morti causati dagli spostamenti tra casa e ufficio. Ciò nonostante ci sono stati anche degli aspetti negativi in questa modalità di lavoro, in particolare la mancanza della comunicazione diretta e del rapporto lavorativo interpersonale con il tutor e gli altri dipendenti dell'azienda, il quale favorisce la crescita personale e l'apprendimento di conoscenze e competenze.

\subsection{Valutazione personale}

Personalmente sono rimasto molto soddisfatto dell'esperienza fatta durante tutto il progetto di stage.\\
Ho potuto lavorare insieme a delle persone molto disponibili, preparate ed entusiaste della loro professione, dalle quali ho imparato molto, sia dal punto di vista delle conoscenze tecniche, ma anche da quello delle competenze professionali; inoltre, nelle occasioni in cui ho potuto lavorare fisicamente in azienda, ho sempre trovato un ambiente confortevole e all'avanguardia, in cui il tempo passava velocemente anche quando il lavoro era difficile.\\
Sono molto contento di aver raggiunto tutti gli obiettivi obbligatori e desiderabili, e di aver realizzato un prodotto valido e utile per l'azienda. Inizialmente avevo qualche dubbio riguardo l'utilizzo pratico della \gls{blockchaing} all'interno del progetto \textit{software}, tuttavia, dopo il periodo di ricerca iniziale e grazie al supporto del tutor e di alcuni colleghi, sia lo scopo che il funzionamento di questa tecnologia mi sono parsi molto più chiari e ragionevoli, consentendomi di lavorare più serenamente e con la consapevolezza della fattibilità del progetto.\\
Sono soddisfatto di aver imparato molte nuove nozioni su svariati ambiti, tecnologie e strumenti; in particolare sulla \gls{blockchaing}, di cui inizialmente avevo un'idea sbagliata e limitata esclusivamente alle criptovalute, ma anche sul linguaggio di programmazione Dart e sul \gls{frameworkg} Flutter, di cui ero completamente all'oscuro e che ho scoperto essere delle ottime alternative a linguaggi più conosciuti e consolidati. Sono rimasto affascinato dalla quantità di possibili applicazioni che queste tecnologie hanno e dalle potenzialità di crescita che avranno nel prossimo futuro.\\
Infine, poter mettere in pratica le conoscenze apprese nel mio percorso di studi universitari in un progetto reale e dall'esito importante è stata una sfida molto interessante e sono contento del risultato finale.
