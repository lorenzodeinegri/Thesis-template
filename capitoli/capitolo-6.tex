% !TEX encoding = UTF-8
% !TEX TS-program = pdflatex
% !TEX root = ../tesi.tex

%**************************************************************
\chapter{Conclusioni}
\label{cap:conclusioni}

%**************************************************************
\section{Consuntivo finale}

Al termine del progetto di stage, le ore di lavoro effettivamente svolte sono risultate in linea con quanto previsto all'interno del piano di lavoro e ammontano ad un totale di 300.\\
Pur rispettando i tempi inizialmente preventivati, si sono verificati degli scostamenti orari, sia in positivo che in negativo, a livello delle singole settimane. In particolare, sebbene sia stato impiegato più tempo del previsto per la formazione iniziale, in quanto il contesto aziendale era completamente nuovo e sconosciuto, la fase di studio e ricerca sulle tecnologie da utilizzare nelle settimane successive è stata svolta più rapidamente, grazie al buon materiale documentale a disposizione. Successivamente, l'implementazione delle funzionalità di base dell'applicazione ha registrato delle discrepanze orarie minime, principalmente dovute al presentarsi o meno di \textit{bug} durante lo sviluppo. La differenza maggiore è stata riscontrata nell'ultimo periodo di realizzazione del \gls{poc}, tuttavia questa non è stata causata da delle problematiche impreviste, bensì dall'investimento di tutto il tempo che era stato risparmiato in precedenza, in modo da riuscire a soddisfare più requisiti possibile dell'applicazione. Le ultime settimane, invece, sono rimaste in linea con quanto preventivato e tutto il lavoro è stato concluso entro i termini previsti.\\
Di seguito viene mostrata la tabella riportante il consuntivo finale del progetto di stage.

\rowcolors{1}{grayer}{white}
\begin{longtable}{|p{5.6cm}|c|c|c|}
	\hline
	\rowcolor{gray}
	\textbf{Attività} & \textbf{Preventivo} & \textbf{Consuntivo} & \textbf{Differenza} \\\hline
	Studio dei concetti fondamentali legati alla \gls{blockchaing} e del contesto aziendale. & 20 & 24 & +4 \\\hline
	Studio approfondito della \gls{blockchaing} di Commerc.io S.r.l. e delle funzionalità da implementare. & 40 & 32 & -8 \\\hline
	Implementazione e verifica delle funzionalità offerte ad un \textit{account} registrato. & 40 & 36 & -4 \\\hline
	Implementazione e verifica delle funzionalità di gestione dell'identità digitale. & 40 & 36 & -4 \\\hline
	Implementazione e verifica delle funzionalità di condivisione di documenti. & 40 & 44 & +4 \\\hline
	Implementazione dell'interfaccia grafica e verifica del prototipo. & 40 & 48 & +8 \\\hline
	Stesura della documentazione ed inizio della relazione. & 40 & 40 & 0 \\\hline
	Termine stesura della relazione e presentazione di quanto svolto in azienda. & 40 & 40 & 0 \\\hline
	
	\caption{Tabella del consuntivo della pianificazione dello stage}
	\label{tab:rendiconto}
\end{longtable}

%**************************************************************
\section{Raggiungimento degli obiettivi}

Al termine del progetto di stage sono stati raggiunti tutti gli obiettivi obbligatori e l'obiettivo desiderabile; quello facoltativo non è stato trattato in quanto, dopo un'attenta valutazione dei tempi necessari per affrontarlo, in rapporto con quelli previsti dal piano di lavoro, è stato deciso di dedicare più lavoro possibile per la realizzazione dell'obiettivo desiderabile, eventualmente andando oltre quanto richiesto inizialmente per lo stesso.

\subsection{Obbligatori}

\begin{itemize}
	\item \textbf{RO-1 raggiunto:} è stato svolto un lavoro di ricerca e di studio riguardo le tematiche relative alla \gls{blockchaing} e alle tecnologie associate, approfondendo poi quelle che sarebbero state utilizzate durante il successivo sviluppo dell'applicazione;
	\item \textbf{RO-2 raggiunto:} è stato implementato un \gls{poc} che interagisce con la \gls{blockchaing} di Commerc.io S.r.l. per la gestione delle identità digitali e la condivisione di documenti firmati;
	\item \textbf{RO-3 raggiunto:} è stata stesa la documentazione relativa a tutti i risultati ottenuti nel corso delle varie fasi del progetto, con particolare riferimento alla gestione e risoluzione delle difficoltà incontrate;
	\item \textbf{RO-4 raggiunto:} è stata stesa una relazione riguardante il lavoro svolto durante tutto il periodo di stage, mettendo in luce le tecnologie e gli strumenti utilizzati, oltre alle procedure di installazione, configurazione ed utilizzo.
\end{itemize}

\subsection{Desiderabili}

\begin{itemize}
	\item \textbf{RD-1 raggiunto:} è stata sviluppata un'applicazione per dispositivi mobili, tramite il \gls{frameworkg} Flutter, che offre un'interfaccia grafica in grado di fornire l'accesso alle funzionalità implementate con il \gls{poc}.
\end{itemize}

\subsection{Facoltativi}

\begin{itemize}
	\item \textbf{RF-1 non raggiunto:} è stato deciso di non iniziare l'implementazione di un'interfaccia \textit{web} per il \gls{poc}, preferendo dedicare tutto il tempo a disposizione all'applicazione per dispositivi mobili.
\end{itemize}

\subsection{Riepilogo}

Di seguito viene mostrata la tabella riportante il resoconto di tutti gli obiettivi con il relativo esito alla data di fine stage.

%**************************************************************
\section{Consegna dei prodotti}

Al termine del progetto di stage sono stati consegnati tutti i prodotti obbligatori e il prodotto desiderabile; quello facoltativo non è stato trattato in quanto, dopo un'attenta valutazione dei tempi necessari per affrontarlo, in rapporto con quelli previsti dal piano di lavoro, è stato deciso di dedicare più lavoro possibile per la realizzazione del prodotto desiderabile, eventualmente andando oltre quanto richiesto inizialmente per lo stesso.

\subsection{Obbligatori}
\begin{itemize}
	\item è stato consegnato il \gls{poc} che mostra il funzionamento della gestione delle identità digitali e del trasferimento di documenti firmati tramite la \gls{blockchaing} \textit{Commercio.network}, sviluppata da Commerc.io S.r.l.;
	\item è stata consegnata la documentazione dettagliata dei risultati ottenuti nel corso delle varie fasi del progetto, con un'analisi critica di tutte le difficoltà incontrate e le relative soluzione adottate;
	\item è stata consegnata una relazione riguardante il lavoro svolto durante tutto il periodo di stage, mettendo in luce le tecnologie e gli strumenti utilizzati, oltre alle procedure di installazione, configurazione ed utilizzo;
	\item è stata consegnata una presentazione che espone il lavoro svolto e le conoscenze apprese, le tecnologie e gli strumenti utilizzati, il prodotto realizzato e il suo funzionamento.
\end{itemize}

\subsection{Desiderabili}

\begin{itemize}
	\item è stata consegnata un'applicazione per dispositivi mobili, realizzata tramite il \gls{frameworkg} Flutter, che offre un'interfaccia grafica in grado di fornire l'accesso alle funzionalità implementate con il \gls{poc}.
\end{itemize}

\subsection{Facoltativi}

\begin{itemize}
	\item non è stata consegnata un'interfaccia \textit{web} per il \gls{poc}, in quanto la sua implementazione è stata esclusa dal progetto di stage, preferendo dedicare tutto il tempo a disposizione all'applicazione per dispositivi mobili.
\end{itemize} 

%**************************************************************
\section{Conoscenze e competenze acquisite}

Lorem

%**************************************************************
\section{Valutazione personale}

Lorem
