% !TEX encoding = UTF-8
% !TEX TS-program = pdflatex
% !TEX root = ../tesi.tex

%**************************************************************
\chapter{Tecnologie e strumenti}
\label{cap:tecnologie-strumenti}

%**************************************************************
\section{Tecnologie}
\subsection*{Commercio.network}

\textit{Commercio.network}\footcite{site:commercio-network} è

\subsection*{Macchina virtuale}

Una macchina virtuale è un \textit{software} che realizza un'emulazione di una macchina fisica attraverso un processo di virtualizzazione, attraverso il quale le risorse \textit{hardware} del dispositivo vengono astratte e rese disponibili al \textit{software} sotto forma virtuale. Di conseguenza non viene fatto accesso e non vengono utilizzate direttamente le risorse fisiche del calcolatore, bensì una loro emulazione logica. Una volta creata una macchina virtuale, su di essa è possibile installare un sistema operativo e delle applicazioni, tramite la stessa procedura che si segue per fare queste operazioni su una macchina fisica.\\
Questa tecnica è usata principalmente per ottimizzare l'utilizzo delle risorse fisiche di un calcolatore, permettendo l'esecuzione contemporanea di più sistemi virtualizzati che svolgono diverse operazioni. Inoltre la virtualizzazione permette di creare sistemi che, essendo slegati dall'\textit{hardware}, non possono in alcun modo causare dei danni al sistema operativo ospitante, anche in caso di corruzione della propria memoria e/o dei propri dischi.\\
Nel progetto di stage è stata utilizzata una macchina virtuale apposita per l'installazione e l'esecuzione di un nodo della rete \textit{Commercio.network}. Così facendo è stato possibile usare la stessa macchina fisica sia per sviluppare e testare il \gls{poc}, all'interno del sistema operativo installato sull'\textit{hardware} reale, sia per interagire con la \gls{blockchaing} di Commerc.io S.r.l., eseguendo tutti i \textit{software} necessari a ciò dentro al sistema virtuale. Solo in questo modo si è potuto mantenere in esecuzione il nodo della rete e, contemporaneamente, avviare ed utilizzare l'applicazione implementata, questa infatti aveva bisogno di accedere alla \gls{blockchaing} per svolgere correttamente tutte le operazioni rese disponibili all'utente.

\subsection*{Git}

Git\footcite{site:git} è

\subsection*{Dart}

Dart\footcite{site:dart} è 

\subsection*{Flutter}

Flutter\footcite{site:flutter} è

%**************************************************************
\section{Strumenti}
\subsection*{VirtualBox}

VirtualBox\footcite{site:virtual-box} è un \textit{software} \gls{open-sourceg} sviluppato da Oracle che fornisce le funzionalità di creazione, controllo ed esecuzione di macchine virtuali ospiti all'interno del sistema operativo del proprio dispositivo fisico. Permette di caricare più sistemi ospiti su un singolo sistema ospitante, ognuno di questi può essere gestito indipendentemente dagli altri, all'interno della propria macchina virtuale. Inoltre, sistema ospitato e sistema ospitante possono comunicare tra di loro sia tramite memoria condivisa che tramite schede di rete virtuali. Infine, supporta sia la virtualizzazione basata su \textit{software} che quella basata su \textit{hardware}, cercando sempre di prediligere quest'ultima alla prima.\\
Nel progetto di stage è stato utilizzato per gestire la macchina virtuale necessaria per l'installazione e l'esecuzione di un nodo della \textit{Commercio.network}.

\subsection*{Ubuntu}

Ubuntu\footcite{site:ubuntu} è un sistema operativo \gls{open-sourceg} basato su Debian, una distribuzione di Linux. Si tratta di un sistema semplice da installare, tramite una procedura guidata, e da utilizzare, attraverso un'intuitiva interfaccia grafica; tuttavia offre molti strumenti utili agli sviluppatori \textit{software}, come gestori di pacchetti, compilatori e ambienti di sviluppo. Inoltre, fa uso di molte tecniche di sicurezza informatica per prevenire il più possibile attacchi sia dall'esterno che dall'interno del sistema. A tal fine prevede una sofisticata gestione dei permessi degli utenti, un controllo totale della connessione e del traffico in rete, e dei compilatori che fanno utilizzo di meccanismi di protezione della memoria. \\
È stato scelto per essere installato all'interno della macchina virtuale utilizzata nel progetto di stage in quanto, tra i sistemi operativi supportati dal \textit{software} che implementa un nodo della rete \textit{Commercio.network}, era quello più adatto ad essere contenuto all'interno dei un ambiente virtuale, sia per la leggerezza in termini di sfruttamento delle risorse, che per la grande capacità di personalizzazione e adattamento.

\subsection*{Cnd}

Cnd è un demone \textit{software}, ossia un programma che, una volta lanciato, resta in esecuzione in \textit{background} senza fornire nessuna possibilità di interazione attiva all'utente. Solitamente questi \textit{software} vengono utilizzati come servizi del sistema operativo e vengono avviati automaticamente durante l'accensione del sistema stesso.\\
Cnd è il \textit{software} sviluppato da Commerc.io S.r.l. per l'implementazione di un nodo della rete \textit{Commercio.network}, è possibile installarlo facilmente seguendo la procedura presente nella documentazione ufficiale\footcite{manual:docs-commercio-network} e, una volta avviato, gestire automaticamente tutte le operazioni necessarie per il corretto funzionamento del nodo e per la corretta interazione con la \gls{blockchaing} in modo da eseguire le operazioni richieste. Oltre a questo, implementa anche un \textit{web server} \gls{rest}\glsfirstoccur{} interrogabile sia dal sistema operativo dove viene eseguito che da dispositivi esterni collegati in rete con esso; è quindi possibile servirsi di questo per operare sulla rete \textit{Commercio.network}.\\
Nel progetto di stage è stato installato all'interno della macchina virtuale Linux per poter poi sfruttarlo per utilizzare la \gls{blockchaing} per la realizzazione delle funzionalità previste per il \gls{poc}. 

\subsection*{Cncli}

Cncli è il software sviluppato da Commerc.io S.r.l. per l'implementazione di un'interfaccia a riga di comando per eseguire operazioni tramite un nodo della rete \textit{Commercio.network} attraverso il demone \textit{software} Cnd. Viene automaticamente installato seguendo la procedura per l'installazione di Cnd e, utilizzandolo dal terminale del sistema operativo, è possibile inserire degli appositi comandi che verranno interpretati e porteranno all'esecuzione di specifiche operazioni sulla \gls{blockchaing} tramite il nodo della rete.\\
Durante il progetto è stato usato, all'interno della macchina virtuale, per testare il funzionamento della \textit{Commercio.network} in base gli scopi dello stage, prima di sviluppare il \gls{poc}. 

\subsection*{Nginx}

Nginx\footcite{site:nginx} è un \textit{software} \gls{open-sourceg} sviluppato da Nginx Inc. che realizza un \textit{web server}. Oltre a questo, può essere anche sfruttato per fare \textit{reverse proxying}, ossia il reindirizzamento delle richieste di un \textit{client} verso uno o più \textit{server} e delle risposte del singolo server verso il \textit{client} opportuno; bilanciamento del carico di richieste dei \textit{client} verso più \textit{server}; \textit{web caching}, ossia la memorizzazione temporanea di pagine \textit{web} recentemente accedute, in modo da reperirle più velocemente per rispondere ad altre richieste successive.\\
Nel progetto di stage è stato utilizzato all'interno della macchina virtuale Linux per esporre all'esterno della stessa le \gls{api} \gls{rest} fornire dal nodo della rete \textit{Commercio.network}; il \textit{server web} eseguito al suo interno, infatti, resta in ascolto per eventuali richieste solamente sull'indirizzo locale del sistema. Tramite la funzionalità di \textit{reverse proxy} di Nginx, quindi, è stato possibile realizzare un reindirizzamento delle richieste fatte all'indirizzo di rete della macchina virtuale, da parte di un dispositivo esterno, verso il \textit{web server} eseguito dal nodo.

\subsection*{GitHub}

GitHub\footcite{site:github} è servizio che permette agli utenti di far ospitare, sulla sua piattaforma \textit{online}, i propri progetti \textit{software}. Fornisce strumenti di sviluppo come il controllo del versionamento distribuito e la gestione del codice sorgente, utilizzando Git, oltre che una serie di altre funzionalità proprie; tra queste si possono trovare sistemi per il controllo d'accesso, per il tracciamento degli errori, per la richiesta di nuove funzionalità e per la gestione dei compiti da svolgere all'interno di un progetto.\\
GitHub offre diversi piani tariffari in base alle funzionalità che si vuole utilizzare, tuttavia, per lo scopo dello stage, è stato sufficiente servirsi del servizio gratuito, in quanto era necessario predisporre un solo progetto privato, senza alcun collaboratore. Infatti la piattaforma è stata utilizzata solamente per effettuare il tracciamento dello sviluppo del codice sorgente del \gls{poc} e il progredire delle sue versioni durante il corso del progetto.

\subsection*{PowerShell}

PowerShell\footcite{site:powershell} è un \gls{frameworkg}\glsfirstoccur{} per l'automatizzazione e la gestione delle configurazioni e dei compiti presenti all'interno del sistema operativo. Consiste di una interfaccia a linea di comando, attraverso la quale l'utente può eseguire diverse operazioni, allo stesso modo che con il terminale di sistema.\\
Nel progetto di stage è stato utilizzato per installare e gestire tutti i \textit{software} necessari per lo sviluppo del \gls{poc}.

\subsection*{Android Studio}

Android Studio\footcite{site:android-studio} è un ambiente di sviluppo software integrato, specificamente progettato per la realizzazione di applicazioni per dispositivi mobili con sistema operativo Android. È stato costruito sulla base di IntelliJ IDEA, un ambiente di sviluppo per i linguaggi Java e Kotlin creato da JetBrains, un'azienda specializzata nella produzione di ambienti di sviluppo integrati per tutti i linguaggi di programmazione più diffusi.\\
Basandosi su IntelliJ, Android Studio ne eredita la maggior parte delle funzionalità, come ad esempio il processo di costruzione dell'applicativo bastato su Gradle, gli strumenti per l'autogenerazione e l'autocorrezione del codice sorgente, e quelli per il rilevamento di problemi di prestazioni, usabilità e compatibilità. Oltre a questo, implementa dei \textit{software} ausiliari per facilitare lo sviluppo delle applicazioni Android, tra questi i più rilevanti sono il generatore di componenti grafici, il visualizzatore dell'anteprima del \textit{layout} dell'interfaccia grafica realizzata e l'emulatore Android che permette di creare un dispositivo virtuale per l'esecuzione e il test delle applicazioni senza necessita di utilizzarne uno fisico.\\
Android Studio è pensato per effettuare delle implementazioni che utilizzano solamente i linguaggi Java o Kotlin, tuttavia mette a disposizione dello sviluppatore una vasta gamma di espansioni che ne espandono le funzionalità. Per il progetto di stage, infatti, è stata prevista la realizzazione di un \gls{poc} in linguaggio Dart tramite il \gls{frameworkg} Flutter; a tal fine sono state installati gli opportuni pacchetti per consentire l'estensione delle funzioni base dell'ambiente di sviluppo anche al codice scritto in Dart/Flutter.
