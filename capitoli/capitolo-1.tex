% !TEX encoding = UTF-8
% !TEX TS-program = pdflatex
% !TEX root = ../tesi.tex

%**************************************************************
\chapter{Introduzione}
\label{cap:introduzione}

%\intro{Introduzione al capitolo}\\

%**************************************************************
\section{L'azienda}

\myCompany{} \companyTitle{}\footcite{site:2c-solution} è un'azienda italiana con sede a Limena (PD) che opera nel settore della digitalizzazione aziendale da circa quindici anni. La sua \textit{mission} è far cessare definitivamente l'utilizzo di documenti cartacei all'interno delle aziende, in modo da semplificare le procedure documentali, agevolando e tracciando il trasferimento delle informazioni; tutto questo nel pieno rispetto delle normative vigenti in materia di \textit{privacy} e sicurezza informatica.

\begin{figure}[!h] 
	\centering 
	\includegraphics[width=0.75\columnwidth]{logo-2c-solution} 
	\caption{Logo di \myCompany{} \companyTitle}
\end{figure}

L'azienda nasce dall'idea di due professionisti dell'\gls{ict}\glsfirstoccur{} di creare un'unica soluzione per permettere alle imprese di eliminare tutta la carta dalla gestione aziendale. Successivamente il prodotto è stato ampliato con diversi moduli, ciascuno dedicato alla realizzazione di una delle tre funzionalità principali:

\begin{itemize}
	\item gestione dei processi aziendali ad alte prestazioni;
	\item organizzazione, catalogazione e ricerca dei documenti tramite una piattaforma digitale semplice ma potente;
	\item creazione di portali accessibili in rete per la condivisione dei documenti aziendali.
\end{itemize}

Col passare del tempo e con l'aumento del numero di regolamentazioni in merito alla gestione digitale delle imprese, l'azienda ha deciso di concentrarsi sulla realizzazione di un nuovo prodotto in grado di consentire la completa gestione documentale a norma di legge. Anche questo \textit{software} è stato diviso in più moduli, in quanto i servizi che offre sono molteplici, in particolare sono implementate le seguenti funzionalità:

\begin{itemize}
	\item firma digitale, archiviazione e conservazione di documenti con valenza legale;
	\item fatturazione elettronica europea, per professionisti, aziende e per la pubblica amministrazione;
	\item gestione di identità digitali personali tramite \gls{spid}\glsfirstoccur{} e \gls{pec}\glsfirstoccur.
\end{itemize}

Fin dall'inizio il lavoro dell'azienda si è basato sui principi che hanno ispirato i due soci fondatori ad iniziare la realizzazione del loro progetto:

\begin{itemize}
	\item l'innovazione tecnologica in ricerca, sviluppo ed utilizzo di soluzioni che facciano uso di strumenti e tecnologie all'avanguardia;
	\item l'aggiornamento continuo dei prodotti sviluppati e la formazione del personale rispetto alle nuove certificazioni e normative vigenti riguardo la gestione della documentazione e dei processi aziendali;
	\item la grande passione per il lavoro svolto e per i frutti che questo porta, fondamentale per svolgerlo al meglio con gioia ed energia.
\end{itemize}

%**************************************************************
\section{Lo stage}

Il principale pilastro che sorregge il lavoro svolto da \myCompany{} \companyTitle{} è la constante ricerca di innovazione e di nuove tecnologie da utilizzare per migliorare i propri prodotti o per offrirne di nuovi ai propri \textit{partner} e clienti. A tal proposito, l'azienda punta molto sulla ricerca di nuove soluzioni ai problemi che affronta quotidianamente e da quest'anno partecipa all'evento StageIT, in convenzione con l'Università di Padova, per presentare progetti di stage a studenti universitari promettenti, con la possibilità di una futura assunzione.\\
Il percorso di stage offerto prevedeva l'inserimento del candidato nel reparto di ricerca e sviluppo, affiancato dal tutor aziendale \myTutor, incaricato di supervisionare e guidare lo stagista durante il lavoro.\\
\myCompany{} \companyTitle{} punta ad inserire la tecnologia \gls{blockchaing}\glsfirstoccur{} nei propri sistemi di gestione dei documenti e delle identità digitali in modo da sfruttarne le proprietà per migliorare le funzionalità già offerte attualmente. A tale scopo l'azienda ha ritenuto che la soluzione migliore fosse l'utilizzo della \gls{blockchaing} realizzata da Commerc.io S.r.l.\footcite{site:commerc-io} (una società affiliata a \myCompany{} \companyTitle), la \textit{Commercio.network}\footcite{site:commercio-network}: infatti questa rete \gls{open-sourceg}\glsfirstoccur{} è stata progettata appositamente per la condivisione di documenti tra le aziende che vi partecipano come nodi. Tuttavia questa non è l'unica funzionalità disponibile: all'interno della \gls{blockchaing} è anche possibile gestire delle identità e delle firme digitali, tutto nel pieno rispetto delle normative europee vigenti in materia.\\
L'obiettivo è quindi realizzare un'applicazione che permetta agli utenti di creare e gestire una propria identità digitale in maniera completamente autonoma e, successivamente, che consenta loro il caricamento e la condivisione di documenti. Tuttavia, prima di procedere al suo sviluppo è necessario analizzare il funzionamento della \textit{commercio.network} per comprenderne pregi e difetti, oltre a capirne le possibili  modalità d'interazione: quali linguaggi di programmazione possono essere utilizzati, quali sono le \gls{api}\glsfirstoccur{} esposte e quali \textit{software} e/o servizi devono essere utilizzati nell'ambiente d'esecuzione.\\
Di conseguenza, tenendo in considerazione che lo sviluppo della \textit{Commercio.network} non è ancora completo, il progetto di stage proposto da \myCompany{} \companyTitle{} si articola in due parti distinte ma allo stesso tempo complementari:

\begin{itemize}
	\item primo periodo: dopo la preliminare formazione sul contesto aziendale, è richiesto lo studio delle tecnologie \gls{blockchaing} da un punto di vista generale, in preparazione all'analisi della rete \textit{Commercio.network} per capirne le caratteristiche e valutarne l'impiego nel prosieguo del progetto aziendale;
	\item secondo periodo: sulla base di quanto appreso nella precedente fase di studio, è richiesto lo sviluppo di un'applicazione con funzione di \gls{poc}\glsfirstoccur{}, per dimostrare la fattibilità del \textit{software} di gestione delle identità digitali e dei documenti associati, utilizzando tutte le tecnologie e gli strumenti necessari per accedere alle funzionalità della \textit{Commercio.network}. 
 \end{itemize}

%**************************************************************
\section{Organizzazione del testo}

\begin{description}
    \item[{\hyperref[cap:stage]{Il secondo capitolo}}] descrive nel dettaglio il progetto di stage, trattando gli obiettivi prefissati isnieme al tutor aziendale, i prodotti attesi dall'azienda al termine del periodo di tirocinio, la pianificazione di tutte le attività lavorative per ciascuna settimana e le motivazioni personali che hanno portato alla scelta di questo progetto tra tutti gli altri.
    
    \item[{\hyperref[cap:blockchain]{Il terzo capitolo}}] approfondisce il tema \gls{blockchaing} da un punto di vista più teorico e di ricerca, delineandone la struttura, spiegandone il funzionamento, mostrandone le caratteristiche e facendo un analisi dei pregi e dei difetti che derivano dal suo utilizzo.
    
    \item[{\hyperref[cap:tecnologie-strumenti]{Il quarto capitolo}}] espone tutte le tecnologie e tutti gli strumenti che è stato necessario utilizzare nelle attività lavorative svolte durante progetto di stage per raggiungerne gli obiettivi preventivati.
    
    \item[{\hyperref[cap:applicazione]{Il quinto capitolo}}] tratta il processo di sviluppo dell'applicazione, realizzata come \gls{poc} del progetto di stage, mettendone in luce le attività di analisi, di progettazione, di codifica e di test, per terminare con delle valutazioni finali sul lavoro svolto e sui prodotti ottenuti.
    
    \item[{\hyperref[cap:conclusioni]{Il sesto capitolo}}] presenta la conclusione del progetto di stage con un consuntivo orario delle attività svolte, la validazione degli obiettivi previsti inizialmente e un'analisi critica di tutte le conoscenze e le competenze acquisite durante il tirocinio, per terminare con una valutazione personale su tutto il percorso di stage.
\end{description}

Per la stesura del documento sono state adottate le seguenti convenzioni tipografiche:

\begin{itemize}
	\item gli acronimi, le abbreviazioni e i termini ambigui o di uso non comune menzionati vengono definiti nel glossario, situato alla fine del presente documento;
	\item per la prima occorrenza dei termini riportati nel glossario viene utilizzata la seguente nomenclatura: \emph{termine}\glsfirstoccur;
	\item i termini in lingua straniera o facenti parti del gergo tecnico sono evidenziati con il carattere \emph{corsivo}.
\end{itemize}
